\documentclass[a4paper,11pt]{article}
\usepackage[T1]{fontenc}
\usepackage[utf8]{inputenc}
\usepackage{lmodern}
\usepackage{amsmath}
\usepackage[top=1.5cm, bottom=2cm, left=2.5cm, right=2.5cm]{geometry}


\title{INFO-F308 : Projet d'année \\ OpenWifiLocalizator (O.W.L.)}
\author{Rémy Detobel, Denis Hoornaert, Nathan Licardo, Robin Petit}

\begin{document}

\maketitle

\section{Introdution}
	Avant d'entrer dans les détails de l'implémentation de notre projet, il est intéressant d'introduire dans les grandes lignes l'objectif de ce dernier.
	Nous avons décidé, de réaliser une application permettant à tout étudiant (ou visiteur) se trouvant sur l'un des campus de l'ULB de pouvoir se diriger au sein de ce dernier.
	L'application permettra de trouver un auditoire sur le campus souhaité. Afin de localiser un étudiant sur le campus nous avons décidé de nous baser sur les bornes wi-fi.
	L'objectif utltime de ce projet sera de présenter notre travail à un large public.
	Il sera donc intéressant de "vulgariser" notre travail et de montrer une application concrète.
	Pour ce faire nous déveloperons une application mobile (cf partie Implémantation).

\section{Système de localisation}
	Comme expliquer dans l'introduction, le système de localisation sera basé sur les points d'accès wifi. Plusieurs techniques de localisation existent.
	Dans le cadre de ce projet nous allons utiliser une \textit{Signal Strength Map} (plus communément noté SSM).
	Le principe est que pour chaque point du graphe, on va identifier et enregistrer l'ensemble des points d'accès couvrant cet endroit.
	Ainsi, quand une personne voudra connaitre sa position, il lui suffira d'identifier les points d'accès l'entourant pour connaitre la point du graphe le plus proche de lui.
	Cette technique peut-être considérée comme une cartographie de la qualité de signal des points d'accès wifi d'un campus.

\section{Implémentation}
	Dans le cadre de ce projet nous avons donc décidé de développer une application permettant aux étudiants "perdus" de se géolocaliser au sein d'un campus directement
	via une application mobile. Le système d'exploitation retenu a été, dans un premier temps, le système Android. Notre choix est du au simple fait que ce système permet un
	accès aisé aux données Wi-fi. Il nous sera donc possible de récupérer les informations des routeurs entourants.

	L'application sera donc développée en Java qui est un langage courant sur Android. Afin d'avoir accès aux données Wi-fi de l'utilisateur (bornes,...) nous avons trouvé
	une API (WifiManager) permettant d'avoir accès à ces informations facilement.
	Nous ne fermons pas la porte à IOS, il est toutefois important de tenir compte de la difficulté d'accè aux données Wi-fi sur cette plate-forme.

\section{Plan de travail}
  \begin{itemize}
	\item Prototype :
	  \begin{enumerate}
	    \item Prise de contacte avec le Département des infrastructures, Service "Projets et Constructions"
	    \item Définir une structure XML
	    \item Implémentation de Dijkstra en java (version PC)
	    \item Implémentation d'un scan des wifi en java (version PC)
	    \item Mise en place d'un programme permettant la maintenance des graphes (Python3)
	  \end{enumerate}
	\item Printemps des sciences:
	  \begin{enumerate}
	    \item Extension du graphe
	    \item Schématiser les plans
	    \item Transposition du prototype sur Android
	  \end{enumerate}
  \end{itemize}


\end{document}
