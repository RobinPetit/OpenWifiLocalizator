\documentclass[11pt,a4paper]{article}

\usepackage[T1]{fontenc}
\usepackage[utf8]{inputenc}
\usepackage[french]{babel} % Global stuff set to french
\usepackage[margin=2.5cm]{geometry} % The margin of the page
\usepackage{graphicx} % to include pictures
\usepackage[hidelinks]{hyperref} % To include hyperlinks in a PDF
\usepackage{appendix} % To make appendixes
\usepackage{color} % For text colors
\usepackage{palatino} % Change font
\usepackage{subcaption}
\usepackage{enumitem}
\usepackage{changepage}
\usepackage{imakeidx}
\usepackage{rotating}
\usepackage{tcolorbox}
\usepackage{amsmath}
\usepackage{tcolorbox}

\graphicspath{{./images/}}

%% Fancy layout
\usepackage{fancyhdr}
\pagestyle{fancy}
\fancyhead[L]{Projet d'année}
\fancyhead[C]{\includegraphics[scale=0.2]{logo.jpg}}
\fancyhead[R]{OpenWifiLocalizator}
% Footer
\fancyfoot[C]{}
\fancyfoot[R]{page \thepage}
% Line
\renewcommand{\headrulewidth}{0.4pt}
\renewcommand{\footrulewidth}{0.4pt}

%opening
\title{}
\author{}

\begin{document}

\topskip0pt
\begin{center}
    \vspace*{\fill}
        \hrule
        \vspace*{2pt}
        \hrule
        \vspace*{15pt}
        \textsc{\Huge{INFO-F308 : Projet d'année \\\vspace*{8pt}
            OWL\\\vspace*{12pt}
            OpenWifiLocalizator}}
        \vspace*{15pt}
        \hrule
        \vspace*{2pt}
        \hrule
  \vspace*{\fill}
\end{center}
\null
\vfill

\large
\hfill Rémy Detobel - \emph{000408013}

\hfill Denis Hoornaert - \emph{000413326}

\hfill Nathan Licardo - \emph{000408278}

25 octobre 2016 \hfill Robin Petit - \emph{000408282}
\newpage



\section{Introdution}
Avant d'entrer dans les détails concernant l'implémentation de notre projet, il est intéressant d'introduire, dans les grandes lignes, l'objectif de réalisation du projet mais également le but final de ce dernier.\\
Nous avons donc décidé, dans le cadre du cours Info-f308, de réaliser une application permettant à tout étudiant (ou visiteur) se trouvant sur l'un des campus de l'ULB de pouvoir se diriger au sein de ce dernier mais également de trouver la direction d'un auditoire se trouvant sur le campus \og courant \fg{} ou de rediriger l'étudiant sur le bon campus dans le cas ou il serait actuellement au mauvais endroit. Afin de pouvoir localiser l'étudiant nous avons opté pour une technique de triangulation. Ce procédé est utilisé dans les GPS et bon nombre de systèmes de localisation. Les détails concernant l'implémentation de notre système seront donnés dans la suite de ce rapport.\\
L'objectif utltime de ce projet sera de présenter notre travail à un large public. Il sera donc intéressant de \og vulgariser \fg{} notre travail et de montrer une application concrète. Pour ce faire nous déveloperons une application mobile (cf partie Implémantation).

\newpage
\section{Etat de l'art}

\section{Plan}

\section{Implémentation}
Dans le cadre de ce projet nous avons donc décidé de développer une application permettant aux étudiants "perdus" de se géolocaliser au sein d'un campus directement via une application mobile. Il nous était donc nécessaire de choisir une plateforme intéressantes pour porter notre application. Le système d'exploitation retenu a été, dans un premier temps, le système Android. Il s'agit en effet d'un OS open-source et extrèmement utilisé ce qui rend notre application facilement accessible pour un large public. L'aspect open-source nous permettra également d'avoir accès de manière aisée aux données utilisateur tel que l'emplacement en temps réel (Android offrant la souplesse adéquate).\\
L'application sera donc développée en Java (utilisé sur Android). Afin de localiser l'utilisateur au sein du campus de manière précise nous avons décidé d'utiliser les bornes wifi comme expliqué précédement. Il nous fallait donc avoir accès aux propriétés des bornes wifi au sein de l'application. Pour ce faire on utilisera l'API "WifiManager" qui nous permet d'obtenir les différentes informations concernant le réseau wifi connecté et les différents réseau accessibles.\\
Nous avons dans un premier temps choisi de développer sur Android. Ce n'est pas pour cela que nous nous fermons la possiblité de porter le programme sur IOS. Il est toutefois important de tenir compte du fait que l'accès aux données de l'utilisateur sont beaucoup plus complèxes sur IOS. L'API "WifiManager" n'existe pas non plus. Le développement sur la plateforme d'Apple fait donc partie des éléments que nous nous résrvons de faire uniquement dans le cas ou l'on aurait assez de temps pour le faire correctement.

\section{Wifi}
  Avant de décrire le fonctionnement et l'origine du Wi-Fi, voici comment le Larousse le définit: ``Réseau local hertzien (sans fil) à haut débit destiné aux liaisons d'équipements informatiques dans un cadre domestique ou professionnel.''.
  Le terme ``Wi-fi'' vient de: ``Wireless Fidelity'' (qui est en oposition à ``Hi-Fi'' qui est généralement associé lui aux appareils sonore).\\
  C'est en 1990 qu'une première connexion est établie entre un ordinateur et un serveur.  Il ne s'agit la que d'une première étape dans le WorldWideWeb.  C'est entre 1997 et 2000 que va réellement se développer le Wifi.  Un standard concernant la technologie sans fil est définit (par les laboratoires Bell et l'Institute of Electrical and Electronics Engineers).\\
  Depuis 2009, de plus en plus d'entreprise propose un service de Wi-Fi gratuitement à leurs client.  On peut le retrouver dans les restaurants, le train, les aeroports, ...
  % Source: http://www.ucopia.com/fr/actualites/lhistoire-du-wi-fi/
  \subsection{Méthodes de localisation}
  Il existe plusieurs méthodes pour se repérer à partir des points d'accès wifi.
  
  \subsubsection{Méthode dites par propagation}
    Cette méthode utilise la trilatération pour déterminer la position actuelle. On peut décomposer cette méthode en plusieurs étapes :
    \begin{enumerate}
      \item On collecte la qualité de signal d'au moins trois points d'accès wifi.
      \item Pour chacun de ces points d'accès, on détermine la distance à celui-ci via une formule prenant en compte la qualité du signal.
      \item À partir de ces distance, on peut "générer" trois cercles et garder un point contenu dans l'intersection de ces cerlces.
    \end{enumerate}
    
    \paragraph{Détermination de la distance}
      La formule utilisée pour la détermination de la distance est la suivante :
      \[d = 10^{\left(\cfrac{27,55-20\log(f)+\left|s\right|}{20}\right)}\]
      Où \begin{itemize}
        \item[] $f$ est la fréquence (généralement 2.4Ghz ou 5.0 GHz)
        \item[] $s$ est la qualité du signal (mesuré en $dBm$)
      \end{itemize}     
    
    \paragraph{Trilateration}
      Soient $x$ et $y$ la position courante de l'utilisateur, $x_{\lambda}$, $y_{\lambda}$ et $r_{\lambda}$ (où $1 \leq \lambda \leq 3$) est le postion et la distance de chaque émetteur.
      \begin{align}
        (x-x_{1})^{2}+(y-y_{1})^{2} = r_{1}^{2} \\
        (x-x_{2})^{2}+(y-y_{2})^{2} = r_{2}^{2} \\
        (x-x_{3})^{2}+(y-y_{3})^{2} = r_{3}^{2}
      \end{align}
      On développe les produits remarquables.
      \begin{align}
        x^{2}-2xx_{1}+x_{1}^{2}+y^{2}-2yy_{1}+y_{1}^{2} = r_{1}^{2} \\
        x^{2}-2xx_{2}+x_{2}^{2}+y^{2}-2yy_{2}+y_{2}^{2} = r_{2}^{2} \\
        x^{2}-2xx_{3}+x_{3}^{2}+y^{2}-2yy_{3}+y_{3}^{2} = r_{3}^{2}
      \end{align}
      On soustrait $(4)$ et $(5)$ ainsi que $(5)$ et $(6)$.
      \begin{center}
        $\left \{
        \begin{array}{c c}
          -2xx_{1}+2xx_{2}+x_{1}^{2}-x_{2}^{2}-2yy_{1}+2yy_{2}+y_{1}^{2}-y_{2}^{2} = r_{1}^{2}-r_{2}^{2} \\
          -2xx_{2}+2xx_{3}+x_{2}^{2}-x_{3}^{2}-2yy_{2}+2yy_{3}+y_{2}^{2}-y_{3}^{2} = r_{2}^{2}-r_{3}^{2}
        \end{array}
        \right.$
      \end{center}
      \begin{center}
        $\left \{
        \begin{array}{c c}
          -2xx_{1}+2xx_{2}-2yy_{1}+2yy_{2} = r_{1}^{2}-r_{2}^{2}-y_{1}^{2}+y_{2}^{2}-x_{1}^{2}+x_{2}^{2} \\
          -2xx_{2}+2xx_{3}-2yy_{2}+2yy_{3} = r_{2}^{2}-r_{3}^{2}-y_{2}^{2}+y_{3}^{2}-x_{2}^{2}+x_{3}^{2}
        \end{array}
        \right.$
      \end{center}
      \begin{center}
        $\left \{
        \begin{array}{c c}
          x(2x_{2}-2x_{1})+y(2y_{2}-2y_{1}) = r_{1}^{2}-r_{2}^{2}-y_{1}^{2}+y_{2}^{2}-x_{1}^{2}+x_{2}^{2} \\
          x(2x_{3}-2x_{2})+y(2y_{3}-2y_{2}) = r_{2}^{2}-r_{3}^{2}-y_{2}^{2}+y_{3}^{2}-x_{2}^{2}+x_{3}^{2}
        \end{array}
        \right.$
      \end{center}
      On remplace ces valeurs par des symbols plus "parlant".
      \begin{center}
        $\left \{
        \begin{array}{c c}
          Ax+By = C \\
          Dx+Ey = F
        \end{array}
        \right.$
      \end{center}
      On developpe pour $x$.
      \begin{align*}
        x &= \frac{C-B(\frac{F-Dx}{E})}{A} \\
        x &= \frac{CE-BF+BDx}{EA} \\
        EAx &= CE-BF+BDx \\
        x &= \frac{CE-BF}{EA-BD}
      \end{align*}
      On developpe pour $y$.
      \begin{align*}
        y &= \frac{F-D(\frac{C-By}{A})}{E} \\
        y &= \frac{FA-DC+DBy}{AE} \\
        AEy &= FA-DC+DBy \\
        y &= \frac{CE-BF}{EA-BD}
      \end{align*}

    \begin{tcolorbox}[title=Avantages :]
      \begin{itemize}
        \item Rapidité de mise en place
      \end{itemize}
    \end{tcolorbox}
    \begin{tcolorbox}[title=Désavantages :]
      \begin{itemize}
        \item Difficulté de mise à jour de la base de donnée.
        \item La qualité du signal peut fortement varié en fonction de :
          \begin{itemize}
            \item la quantité de personnes présentent sur le point d'accès.
            \item divers éléments externe tels que des murs, ...
          \end{itemize}
      \end{itemize}
    \end{tcolorbox}

\end{document}
