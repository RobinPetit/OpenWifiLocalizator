\documentclass[11pt,a4paper]{article}

\usepackage[T1]{fontenc}
\usepackage[utf8]{inputenc}
\usepackage[french]{babel} % Global stuff set to french
\usepackage[margin=2.5cm]{geometry} % The margin of the page
\usepackage{graphicx} % to include pictures
\usepackage[hidelinks]{hyperref} % To include hyperlinks in a PDF
\usepackage{appendix} % To make appendixes
\usepackage{color} % For text colors
\usepackage{palatino} % Change font
\usepackage{subcaption}
\usepackage{enumitem}
\usepackage{changepage}
\usepackage{imakeidx}
\usepackage{rotating}
\usepackage{tcolorbox}
\usepackage{amsmath, amssymb,  commath}
\usepackage{tcolorbox}
\usepackage{hyperref}

\graphicspath{{./images/}}

%% Fancy layout
\usepackage{fancyhdr}
\pagestyle{fancy}
\fancyhead[L]{Projet d'année}
\fancyhead[C]{\includegraphics[scale=0.2]{logo.jpg}}
\fancyhead[R]{OpenWifiLocalizator}
% Footer
\fancyfoot[C]{}
\fancyfoot[R]{page \thepage}
% Line
\renewcommand{\headrulewidth}{0.4pt}
\renewcommand{\footrulewidth}{0.4pt}

%opening
\title{}
\author{}

\begin{document}

\topskip0pt
\begin{center}
    \vspace*{\fill}
        \hrule
        \vspace*{2pt}
        \hrule
        \vspace*{15pt}
        \textsc{\Huge{INFO-F308 : Projet d'année \\\vspace*{8pt}
            OWL\\\vspace*{12pt}
            OpenWifiLocalizator}}
        \vspace*{15pt}
        \hrule
        \vspace*{2pt}
        \hrule
  \vspace*{\fill}
\end{center}
\null
\vfill

\large
\hfill Rémy Detobel - \emph{000408013}

\hfill Denis Hoornaert - \emph{000413326}

\hfill Nathan Licardo - \emph{000408278}

25 octobre 2016 \hfill Robin Petit - \emph{000408282}
\newpage

\tableofcontents

\newpage

\section{Introdution}
Avant d'entrer dans les détails concernant l'implémentation de notre projet, il est intéressant d'introduire, dans les grandes lignes, l'objectif de réalisation du projet mais également le but final de ce dernier.\\
Nous avons donc décidé, dans le cadre du cours Info-f308, de réaliser une application permettant à tout étudiant (ou visiteur) se trouvant sur l'un des campus de l'ULB de pouvoir se diriger au sein de ce dernier. L'idée principale de notre projet est de pouvoir rediriger un nouvel étudiant (ou un individu ne connaissant pas le campus sur lequel il se trouve) vers un auditoire cible. Afin de pouvoir localiser avec précision l'individu en question, nous sommes parti de l'idée de la triangulation. Cette technique est notamment utilisée dans les GPS. Les détails concernant l'implémentation de notre système seront donnés dans la suite de ce rapport.\\
L'objectif ultime de ce projet sera de présenter notre travail à un large public. Il sera donc intéressant de \og vulgariser \fg{} notre programme et de montrer une application concrète. Pour ce faire nous développerons une application mobile (cf partie Implémantation).\\\\
Afin de réaliser notre projet nous avons eu besoin d'informations précises concernant l'ULB. Nous tenons donc à remercier M. Bruno Delcroix qui nous à permis d'accéder aux plans des campus concernés et donc de mettre en place notre application.


\newpage
\section{Implémentation}
Dans le cadre de ce projet nous avons donc décidé de développer une application permettant aux étudiants i"perdus" de se géolocaliser au sein d'un campus directement via une application mobile. Il nous était donc nécessaire de choisir une plateforme intéressantes pour porter notre application. Le système d'exploitation retenu a été, dans un premier temps, le système Android. La raison de notre choix repose principalement sur l'aspect \og~ouvert~\fg{} de la plate-forme. Il nous est en effet possible d'accéder aux données liées aux réseaux Wi-Fi.\\
L'application sera donc développée en Java (langage Android). Afin de localiser l'utilisateur au sein du campus de manière précise nous avons décidé d'utiliser les bornes Wi-Fi comme expliqué précédemment. Il nous fallait donc avoir accès aux propriétés des bornes Wi-Fi au sein de l'application. Pour ce faire on utilisera l'API \textit{WifiManager} qui nous permet d'obtenir les différentes informations concernant les réseaux Wi-Fi connectés et les différents réseaux accessibles.\\
Nous avons dans un premier temps choisi de développer sur Android. Ce n'est pas pour cela que nous nous fermons la possibilité de porter le programme sur iOS. Il est toutefois important de tenir compte du fait que l'accès aux données de l'utilisateur sont beaucoup plus complexes et limitées sur iOS. L'API \textit{WifiManager} n'existe pas non plus, et nous n'en avons pas d'équivalent. Le développement sur la plateforme d'Apple fait donc partie des éléments que nous nous réservons de faire uniquement dans le cas ou l'on aurait assez de temps pour le faire correctement.

\section{Wi-Fi}
  Avant de décrire le fonctionnement et l'origine du Wi-Fi, voici comment le Larousse le définit~: «~Réseau local hertzien (sans fil) à haut débit destiné aux liaisons d'équipements informatiques dans un cadre domestique ou professionnel.~»
  Le terme «~Wi-Fi~» vient de: «~Wireless Fidelity~» (qui est en opposition à «~Hi-Fi~» qui est généralement associé lui aux appareils sonores).\\
  C'est en 1990 qu'une première connexion est établie entre un ordinateur et un serveur.  Il ne s'agit la que d'une première étape dans le \textit{World Wide Web}. C'est entre 1997 et 2000 que va réellement se développer le Wi-Fi. Un standard concernant la technologie sans fil est définit (par les laboratoires Bell et l'\textit{Institute of Electrical and Electronics Engineers}).\\
  Depuis 2009, de plus en plus d'entreprise proposent un service de Wi-Fi gratuitement à leurs clients.  On peut le retrouver dans les restaurants, les trains, les aéroports, etc.
  % Source: http://www.ucopia.com/fr/actualites/lhistoire-du-wi-fi/

\section{Modélisation des infrasctructures}
  \subsection{Recherche du plus court chemin}
    \subsubsection{Dijkstra}
    \subsubsection{A*}

\section{Méthodes de localisation}
  Il existe plusieurs méthodes pour se repérer à partir des points d'accès Wi-Fi.

  \subsection{Méthode dites par propagation}
    Cette méthode utilise la trilatération pour déterminer la position actuelle. On peut décomposer cette méthode en plusieurs étapes~:
    \begin{enumerate}
      \item on collecte la qualité de signal d'au moins trois points d'accès wifi~;
      \item pour chacun de ces points d'accès, on détermine la distance à celui-ci via une formule prenant en compte la qualité du signal~;
      \item à partir de ces distances, on peut «~générer~» trois cercles et garder un point contenu dans l'intersection de ces cercles\footnote{Mathématiquement, trois cercles s'intersectent en un et un seul point. Cependant, toutes interférences prises en comptes, on peut visualiser les cercles comme ayant une \textit{épaisseur} correspondant à l'incertitude. On peut alors parler d'\textit{un point de l'intersection des trois cercles.}} via une méthode appelée \textit{trilatération}.
    \end{enumerate}

    \subsubsection{Trilateration}
      Le principe de la trilatération est de trouver un point du plan se situant dans l'intersection de plusieurs cercles. Dans notre cas, les cerles représentent les régions de l'espace couvertent pas les \textit{Wifi} dont le centre est la position du points d'accès \textit{Wifi} et où le point se trouvant à l'intersection des cercles représente la position de l'utilisateur.
      \begin{center}
        \includegraphics[scale=0.8]{trilateration.png}\\
        \textit{Simple exemple de trilatération}
      \end{center}
      De manière formelle, la détermination de la position de l'utilisateur se fait suivant le raisonnement suivant :\\\\
      Soient $(x, y) \in \mathbb R^2$ la position actuelle de l'utilisateur, $x_{\lambda}$, $y_{\lambda}$ et $r_{\lambda}$ (où $1 \leq \lambda \leq 3$) est le position et la distance de chaque
    émetteur. Alors on peut écrire~:
      \begin{align}
        (x-x_{1})^{2}+(y-y_{1})^{2} = r_{1}^{2}, \\
        (x-x_{2})^{2}+(y-y_{2})^{2} = r_{2}^{2}, \\
        (x-x_{3})^{2}+(y-y_{3})^{2} = r_{3}^{2}.
      \end{align}
      On développe les produits remarquables~:
      \begin{align}
        x^{2}-2xx_{1}+x_{1}^{2}+y^{2}-2yy_{1}+y_{1}^{2} = r_{1}^{2} \\
        x^{2}-2xx_{2}+x_{2}^{2}+y^{2}-2yy_{2}+y_{2}^{2} = r_{2}^{2} \\
        x^{2}-2xx_{3}+x_{3}^{2}+y^{2}-2yy_{3}+y_{3}^{2} = r_{3}^{2}
      \end{align}
      On soustrait $(4)$ à $(5)$ ainsi que $(5)$ à $(6)$~:
      \begin{center}
        $\left \{
        \begin{array}{c c}
          -2xx_{1}+2xx_{2}+x_{1}^{2}-x_{2}^{2}-2yy_{1}+2yy_{2}+y_{1}^{2}-y_{2}^{2} = r_{1}^{2}-r_{2}^{2} \\
          -2xx_{2}+2xx_{3}+x_{2}^{2}-x_{3}^{2}-2yy_{2}+2yy_{3}+y_{2}^{2}-y_{3}^{2} = r_{2}^{2}-r_{3}^{2}
        \end{array}
        \right.$
      \end{center}
      \begin{center}
        $\left \{
        \begin{array}{c c}
          -2xx_{1}+2xx_{2}-2yy_{1}+2yy_{2} = r_{1}^{2}-r_{2}^{2}-y_{1}^{2}+y_{2}^{2}-x_{1}^{2}+x_{2}^{2} \\
          -2xx_{2}+2xx_{3}-2yy_{2}+2yy_{3} = r_{2}^{2}-r_{3}^{2}-y_{2}^{2}+y_{3}^{2}-x_{2}^{2}+x_{3}^{2}
        \end{array}
        \right.$
      \end{center}
      \begin{center}
        $\left \{
        \begin{array}{c c}
          x(2x_{2}-2x_{1})+y(2y_{2}-2y_{1}) = r_{1}^{2}-r_{2}^{2}-y_{1}^{2}+y_{2}^{2}-x_{1}^{2}+x_{2}^{2} \\
          x(2x_{3}-2x_{2})+y(2y_{3}-2y_{2}) = r_{2}^{2}-r_{3}^{2}-y_{2}^{2}+y_{3}^{2}-x_{2}^{2}+x_{3}^{2}
        \end{array}
        \right.$
      \end{center}
      On remplace ces valeurs constantes par des symboles plus \textit{parlant}.
      \begin{center}
        $\left \{
        \begin{array}{c c}
          Ax+By = C \\
          Dx+Ey = F
        \end{array}
        \right.$
      \end{center}
      On développe d'abord pour $x$~:
      \begin{align*}
        x &= \frac{C-B(\frac{F-Dx}{E})}{A} \\
        x &= \frac{CE-BF+BDx}{EA} \\
        EAx &= CE-BF+BDx \\
        x &= \frac{CE-BF}{EA-BD}
      \end{align*}
      On développe ensuite pour $y$~:
      \begin{align*}
        y &= \frac{F-D(\frac{C-By}{A})}{E} \\
        y &= \frac{FA-DC+DBy}{AE} \\
        AEy &= FA-DC+DBy \\
        y &= \frac{FA-DC}{EA-BD}
      \end{align*}

    \subsubsection{Détermination de la distance}
      La formule utilisée pour la détermination de la distance est la suivante~:
      \[d_f(s) = 10^{\alpha_f(s)},\]
      où~:
      \[\alpha_f(s) = \frac{27,55-20\log_{10}(f)+\abs s}{20},\]
        avec~:
      \begin{itemize}
        \item[] $f$ est la fréquence (généralement 2.4Ghz ou 5.0 GHz)~;
        \item[] $s$ est la qualité du signal (mesuré en $dBm$).
      \end{itemize}
      \begin{center}
        \includegraphics[scale=0.5]{signal-propagation.jpg}\\
        \textit{Répartition de la qualité de signal en fonction de la distance.}
      \end{center}
      \begin{center}
        \includegraphics[scale=0.4]{wifi-propagation.png}\\
        \textit{Représentation de la qualité de signal au sein d'un batiment.}
      \end{center}

    \begin{tcolorbox}[title=Avantages :]
      \begin{itemize}
        \item Rapidité de mise en place.
      \end{itemize}
    \end{tcolorbox}
    \begin{tcolorbox}[title=Désavantages :]
      \begin{itemize}
        \item Difficulté de mise à jour de la base de donnée~;
        \item la qualité du signal peut fortement varier en fonction de~:
          \begin{itemize}
            \item la quantité de personnes présentent sur le point d'accès~;
            \item divers éléments externe tels que des murs, etc.
        \end{itemize}
      \end{itemize}
    \end{tcolorbox}
  \subsection{Méthode dites de ??? (SSM)}
    Une autre méthode de localisation utilise une \textit{signal strength map} (notée \textit{SSM}). Une \textit{SSM} est défini comme étant une cartographie des régions couvertes par les signaux \textit{Wifi}. La construction de cette carte se fait en regardant et enregistrant quels sont les points d'accès \textit{Wifi} que l'on capte pour à une certaine position. On va donc enregistrer, dans une base de données, pour chaque position désirée, les identifiants (\textit{BSS}) et les qualités de signal des points d'accès \textit{Wifi} couvrant cette position.
    \subsubsection{Identification de la position}
      La base de données va, comme mentionné ci-dessus, contenir pour chaque identifiant un vecteur de points d'accès \textit{Wifi}. On peut schématiser cette base de données comme suit : 
      %\includegraphics[scale=0.5]{}
      Le problème à résoudre est donc de trouver le vecteur au sein de la base de données qui est le plus proche du vecteur obtenu par l'utilisateur. Ce genre de problème est assez courant dans certain secteur des sciences informatiques tel que la bioinformatique. En effet, dans la cadre de la bioinformatique, de nombreux algorithmes permettant de déterminer la similarité entre deux séquences ont été mis au point (par exemple : Needlman-Wunsch). Cependant, dans notre cas, nous ne pourrons pas directement utiliser ce genre d'algorithme car les élément que nous désirons analyser présentent des aspects différents de ceux des séquence d'acides aminés.
      \paragraph{Fonction de similarité}
        \begin{center}
          \includegraphics[scale=0.5]{similarity-function.jpg}
        \end{center}
        \begin{center}
          \includegraphics[scale=0.5]{similarity-function-profil.jpg}
        \end{center}
    

\section{Traitement du graphe}
	\subsection{Utilisation de la triangulation}
		Le graphe manipulé sera représenté par $\Gamma = (V, E)$, avec $V$ l'ensemble des locaux accessibles ainsi que l'ensemble des points additionnels dans les couloirs
		afin de relier les locaux entre eux, et $E$ l'ensemble des liens permettant de joindre les nœuds.

		L'idée initiale était de lier le graphe des nœuds avec un graphe des bornes Wi-Fi et de diminuer les poids des arêtes d'une constante $K \gneqq 0$ pour les arêtes
		ayant une borne à une des deux extrémités. Ce cas aurait laissé la possibilité d'avoir des arêtes de poids négatif, et donc un algorithme classique tel que Dijkstra
		ou A* n'aurait pas été possible.

		Nous avions donc pensé qu'un algorithme plus général tel que \textit{Bellman-Ford} serait nécessaire (afin de gérer les poids négatifs), et pour pouvoir diminuer
		les calculs, faire une passe de \textit{Floyd-Warshall} au préalable afin de pouvoir élaguer tous les chemins en cours de calcul, de poids supérieur à la borne
		donnée par Floyd-Warshall.

	\subsection{Utilisation de la SSM}
		Cependant, le fait d'utiliser une SSM rend la localisation plus simple~: il n'est plus nécessaire de maintenir un graphe des bornes, et donc il n'y a plus
		qu'un seul graphe, à savoir celui des locaux (dans lequel seront stockées les informations relatives aux réseaux). Cela retire la nécessité de retirer une
		constante $K$ et donc, il n'y a plus de risque d'avoir des arêtes de poids négatif.

		Dès lors, un algorithme standard tel que Dijkstra ou A* peut être utilisé. C'est A* qui sera implémenté ici car il est de complexité $O(\abs E)$ dans
		le pire des cas, alors que Dijkstra est de complexité $O(\abs E + \abs V\log\abs V)$ dans le pire des cas.

		Le gain est permis par l'utilisation d'une heuristique qui nous suffira amplement ici.

\section{Base de données}
  \subsection{"Flat files"}
  \subsection{SQL}

\section{Conclusion}

\section{Remerciements}
  Payot du service architecture

\section{Bibliographie}
  \begin{thebibliography}{9}
    \bibitem{ssm}
      Matteo Cypriani, Frédéric Lassabe, Philippe Canalda, François Spies,
      \emph{Wi-Fi-Based Indoor Positioning: Basic Techniques, Hybrid Algorithms and Open Software Platform}.
      2010.
    \bibitem{Roumanie}
      Bianca BOBESCU, Marian ALEXANDRU
      \emph{Mobile indoor positioning using Wi-fi localisation}.
      Transilvania University, Brasov, Romania,
      2015.
    \bibitem{trilateration}
      OnkarPathak, Pratik Palaskar, Rajesh Palkar, Mayur Tawari,
      \emph{Wi-Fi Indoor Positioning System based on RSSI Measurements from Wi-Fi Access Points A Trilateration Approach}.
      International Journal of Scientific \& Engineering Research,
      2014.
    \bibitem{GaussianProcessesFerris}
      Brian Ferris, Dirk Hähnel, Dieter Fox,
      \emph{Gaussian Processes for Signal Strength-Based Location Estimation}.
      University of Washington, Department of Computer Science \& Engineering, Seattle, WA Intel Research Seattle, Seattle, WA.
  \end{thebibliography}

  %---------- SCANNING ----------
  %# www.mdpi.com/1424-8220/15/9/21824/pdf
  %https://www.researchgate.net/publication/224198838_Wi-Fi-based_indoor_positioning_Basic_techniques_hybrid_algorithms_and_open_software_platform
  %x https://fruct.org/publications/abstract16/files/Shc1.pdf
  %http://www.afahc.ro/ro/revista/2015_1/119.pdf
  %x http://file.scirp.org/pdf/CN_2013071010352139.pdf
  %http://www.ijser.org/researchpaper%5CWi-Fi-Indoor-Positioning-System-Based-on-RSSI-Measurements.pdf
  %https://www.researchgate.net/profile/Suhailan_Safei/publication/230771403_INDOOR_POSITION_DETECTION_USING_WIFI_AND_TRILATERATION_TECHNIQUE/links/5513e9120cf2eda0df3031f0.pdf
  %http://www.ee.ucl.ac.uk/lcs/previous/LCS2005/12.pdf
  %http://www.int-arch-photogramm-remote-sens-spatial-inf-sci.net/XXXVIII-4-C26/1/2012/isprsarchives-XXXVIII-4-C26-1-2012.pdf
  %---------- LOCALISATION ----------
  %http://www.roboticsproceedings.org/rss02/p39.pdf
  %https://venturi.fbk.eu/wp-content/uploads/2011/10/AraMes_WIMOB_2014.pdf
  %http://www.tik.ee.ethz.ch/file/2490a7adb6a163b9c5be1510d033870a/sawn05.pdf
  %https://felixduvallet.github.io/pubs/2008-WiFi-IROS.pdf
  %http://www.cs.cmu.edu/~mmv/papers/10icra-joydeep.pdf
  %https://papers.nips.cc/paper/2541-gpps-a-gaussian-process-positioning-system-for-cellular-networks.pdf
  %http://www-cs.stanford.edu/people/dstavens/icra11/huang_etal_icra11.pdf
  %https://www.ncbi.nlm.nih.gov/pmc/articles/PMC5017359/
  %http://www.robot.t.u-tokyo.ac.jp/~yamashita/paper/E/E293Final.pdf

\section{Annexes}

\end{document}
